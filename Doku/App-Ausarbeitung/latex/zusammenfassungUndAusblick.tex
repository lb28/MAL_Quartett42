\chapter{Zusammenfassung und Ausblick}
\label{cha:zusammenfassungUndAusblick}

In diesem Kapitel soll ein kurzes Fazit zur App-Entwicklung und zum Anwendungsfach allgemein festgehalten werden. 

Auch wenn der Einstieg in die App-Entwicklung mit dem ``EiTimer'' nicht ganz leicht war, haben wir dort schon gemerkt, dass uns die Entwicklung liegt und Spaß bereitet. Wir haben gerade zu Beginn bei der Implementierung der ``EiTimer''-Anwendung und später unserer Quartett42-App vieles gelernt, was wir im zweiten Teil des Anwendungsfachs natürlich anwenden und weiter vertiefen können.

In unseren Augen ist uns die Anwendung größtenteils gut gelungen. Die Qualität der Anwendung wäre theoretisch ausreichend für eine Veröffentlichung im App Store, da es gerade bei Quartett-Spielen eine Marktlücke gibt.\\ Natürlich gäbe es noch einige Stellen, an der sie weiter verbessert werden könnte, wie beispielsweise das Design oder weitere naheliegende Funktionen (Spieler vs. Spieler, Online/Bluetooth-Spiel, ...).\\ Ein nicht zu unterschätzender Faktor war hierbei die Zeit, die neben den anderen Vorlesungen, Prüfungen, etc. limitiert war. Uns ist jedoch im gegebenen Zeitrahmen ein abgerundetes Produkt ohne ``unfertige'' Funktionen gelungen.

Da wir beim ersten Teil des Projekts gut als Team zusammenarbeiten konnten, haben wir beschlossen auch im zweiten Teil des Anwendungsfachs wieder gemeinsam eine Anwendung zu entwickeln, für die wir auch schon einige Ideen gesammelt haben. Im zweiten Teil des Projekts werden wir versuchen, weitere Wissenslücken in der Android-Entwicklung zu schließen.