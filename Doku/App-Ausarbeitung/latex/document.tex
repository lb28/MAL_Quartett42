\documentclass[numbers=noenddot]{thesis}

% hier namen etc. einsetzen
\fullname{Fabian Fischbach, Luis Beaucamp und Tim Stenzel}
%\email{vorname.nachname@uni-ulm.de}
\headline{Mobile Application Lab}
\titel{Thema}
\jahr{2017}
\matnr{Matrikel-Nr}
\gutachterA{Marc Schickler}
%\gutachterB{Gutachter 2}
\betreuer{Marc Schickler}
\typ{Ausarbeitung zur App }
\fakultaet{Ingenieurwissenschaften, Informatik und \\Psychologie}
\institut{Institut für Datenbanken und Informationssysteme}

% Falls keine Lizenz gewünscht wird bitte den folgenden Text entfernen.
% Die Lizenz erlaubt es zu nichtkommerziellen Zwecken die Arbeit zu
% vervielfältigen und Kopien zu machen. Dabei muss aber immer der Autor
% angegeben werden. Eine kommerzielle Verwertung ist für den Autor
% weiter möglich.
\license{
This work is licensed under the Creative Commons.
Attribution-NonCommercial-ShareAlike 3.0 License. To view a copy of this
license, visit http://creativecommons.org/licenses/by-nc-sa/3.0/de/ or send a
letter to Creative Commons, 543 Howard Street, 5th Floor, San Francisco,
California, 94105, USA. \\ Satz: PDF-\LaTeXe
}

\hypersetup{%
	pdftitle=\pdfescapestring{\thetitel},
	pdfauthor={\thefullname},
 	pdfsubject={\thetyp},
}


% trennungsregeln
\hyphenation{Sil-ben-trenn-ung}

%-----------Pakete einfügen----------------
\usepackage{mwe}
\renewcaptionname{ngerman}{\figurename}{Abb.}


%------------------------------------------

\begin{document}
\frontmatter
\maketitle
% impressum
\clearpage
\impressum

\cleardoublepage
% ab hier zeilenabstand 1,4 fach 10pt/14pt
\setstretch{1.4}

%\section*{Kurzfassung}
Die Kurzfassung (engl. Abstract) einer Abschlussarbeit enthält zwei Blöcke. Der
erste Block enthält eine  kurze Hinführung/Motivation zum Thema sowie einer
anschließenden Beschreibung der Problemstellung (ca. 5-8 Sätze). Der zweite
Block der Kursfassung gibt die Zielsetzung bzw. den Beitrag der Abschlussarbeit
wieder (ebenfalls ca. 5-8 Sätze).

===========================================

ChangeLog:

2015-10-12: Hacks für Literaturverzeichnis eingebaut. Kommentare in der BibTex
Datei beachten!

2015-07-21: Fakultätname angepasst (+ Psychologie)



%\cleardoublepage
%\section*{Danksagung}
An dieser Stelle erfolgt die Danksagung an Personen, die einen bei der Erstellung der Abschlussarbeit unterstützt haben.

% inhaltsverzeichnis einfügen
\tableofcontents

\mainmatter
% hier kommen die kapitel der arbeit
\chapter{Einleitung}
\label{cha:einleitung}

In dieser Dokumentation wird die Entwicklung einer App im Rahmen des Anwendungsfaches Mobile Application Lab an der Universität Ulm und die dadurch erstellte App in Form einer Quartett App vorgestellt.

% Abschnitt: Problemstellung
\section{Motivation und Problemstellung}
\label{sec:einleitung:problemstellung}

Die Problemstellung wurde uns im Rahmen dieses Projektes schon gegeben, da wir uns auf die Entwicklung einer Quartett App für Smartphones konzentrieren sollten. Das Prinzip des beliebten Kartenspiels soll für ein Smartphone entwickelt werden und so zu jeder Zeit und an jedem Ort auch ganz ohne physiche Karten spielbar sein.

Beim Anschauen des aktuellen Marktes für Quartett Apps fällt schnell die Vielzahl an verschiedenen Apps auf. Diese erweisen nach genauerem Betrachten jedoch erhebliche Mängel auf. So sind manche von ihnen sehr veraltet und funktionieren nicht mehr richtig auf neueren Smartphone Modellen. Auch entsprechen diese inhaltlich nicht unseren Vorstellungen von einer guten Quartett App. Sie sind sehr beschränkt, was die verschiedenen Spielmodi angeht und beschräken sich meistens auf ein einziges Kartendeck oder Decks aus einem Themengebiet.

% Abschnitt: Zielsetzung
\section{Zielsetzung}
\label{sec:einleitung:zielsetzung}

Da auf dem Markt eine Nachfrage besteht wollen wir diese ausnutzen und eine eigene Quartett Anwendung erstellen. Diese wollen wir auf Basis von Android und Java entwickeln. Dabei geht es uns primär darum, den Umgang mit den neuen Techniken zu erlernen und eine Grunderfahrung im Programmieren von Android Anwendungen zu erlangen, sodass wir diese nach Abschließen des Projektes beherrschen können.

Inhaltlich möchten wir eine Quartett App entwickeln, die sich als Singleplayer wie ein richtiges Quartett spielen lässt. Sie soll verschiedene Spielmodi haben, welche frei konfigurierbar sein sollen. Zudem soll die Auswahl an Decks breit gefächert sein, was durch einen Deckcreator und einer Onlinefunktion zum Up- und Downloaden realisiert werden soll. Der Anwendung soll zudem die Möglichkeit bieten, alle Karten anzugucken und laufende Spiele zu unterbrechen. Dabei soll die App benutzerfreundlich sein und schön aussehen, sowie auf den neusten aber auch auf älteren Android Smartphones lauffähig sein.

% Abschnitt: Struktur der Arbeit
\section{Struktur der Arbeit}
\label{sec:einleitung:struktur}

In dieser Dokumentation werden zuerst einmal grundlegend die Quartett Spielregeln erklärt sowie Android vorgestellt und unsere verwendeten Frameworks präsentiert. Den Hauptteil bildet die Implementierung, welche unsere App im Allgemeinen und mit ihren Besonderheiten vorstellt. Außerdem wird darin unsere Architektur gezeigt und wir erleutern Schwierigkeiten, die wir während der Implementierungsphase hatten. Abschließend gibt es einen Anforderungsabgleich und einen Ausblick auf die Zukunft des Projektes.\\

\begin{figure}[htp]
	\centering
  	\includegraphics[width=0.4\textwidth]{img/quartett42_logo.png}
	\caption{Quartett42 Logo}
	\label{figure:quartett42logo}
\end{figure}

\chapter{Grundlagen}
\label{cha:grundlagen}

% Abschnitt: Quartettspiel
\section{Quartettspiel}
\label{sec:grundlagen:quartettspiel}

Zum Quartettspielen sind natürlich einige Regeln notwendig, die im Folgenden erklärt werden. \\
Gespielt wird Eins gegen Eins, Spieler gegen Computer. Zuerst wählt der Spieler den Schwierigkeitsgrad (leicht, mittel oder schwer) des Computers, dann den Spielmodus und das Limit für das Spielende (Zeit-, Runden- oder Punktemodus und entsprechend die Spielzeit, Runden- oder Punkteanzahl). Danach wird ein Deck gewählt und gemischt. Computer und Spieler bekommen jeweils die Hälfte der Karten verdeckt auf einem Stapel, bei dem immer nur die oberste Karte sichtbar ist, und es wird zufällig bestimmt wer mit dem ersten Zug beginnen darf. Ein Zug läuft im Grunde genauso ab wie beim ``echten'' Quartett: Der Spieler, der am Zug ist, wählt ein Attribut (Beispiel in einem Autoquartett: Höchstgeschwindigkeit) und nennt den entsprechenden Wert. Der andere Spieler gibt nun ebenfalls seinen Wert bei dem gewählten Attribut bekannt (Beispiel von oben: Höchstgeschwindigkeit) und die Werte werden verglichen. Für jedes Attribut wurde vor dem Spiel festgelegt ob für dieses ein höherer oder niedrigerer Wert gewinnt. Darauf basierend wird nun der Vergleich durchgeführt. Der Spieler mit dem besseren Wert gewinnt den Vergleich, bekommt beide Karten unter seinen Stapel und darf im nächsten Vergleich das Attribut wählen. Bei einem Unentschieden behält jeder seine Karte, legt sie unter seinen Stapel und der Spieler, der das Attribut gewählt hat, wählt auch das nächste. Das Spiel ist zu Ende, wenn ein Spieler keine Karten mehr auf seinem Stapel hat, oder wenn das Limit des gewählten Spielmodus erreicht ist (Zeit abgelaufen / alle Runden ausgespielt / Punktelimit erreicht).

Dies sind die normalen Regeln und Spielmodi für ein Quartettspiel. In unserer App gibt es aber zusätzlich neben dem normalen Modus (bei dem jeweils, wie festgelegt, der höhere oder niedrigere Wert den Vergleich gewinnt) auch noch den sog. ``Insane-Modus'', bei dem jeweils nicht der vorher festgelegte höhere oder niedrigere Wert gewinnt, sondern genau umgekehrt. Damit es im Spiel dennoch nicht zu Verwirrungen kommt, ist neben jedem Attribut ein Pfeil, der angibt, ob (im aktuellen Modus) ein höherer oder niedrigerer Wert beim Vergleich gewinnt.
Unabhängig vom Insane-Modus kann der Benutzer zusätzlich entscheiden, ob er den Expertenmodus spielen möchte oder nicht. Der Expertenmodus ist jedoch nichts für Anfänger, denn es werden die Werte einer Karte ``zensiert'' (durch ein '?' ersetzt), sodass man das Deck bzw. die Karten schon ein bisschen besser kennen muss, um hier erfolgreich zu sein.


% Abschnitt: Mobile Plattform
\section{Mobile Plattform}
\label{sec:grundlagen:plattforml}

Android ist ein mobiles Betriebssystem, also für Smartphones und Tablets, das von Google entwickelt wurde und auf Linux basiert. Die App-Entwicklung ist geprägt durch einzelne Aktivitäten (eine Aktivität ist eine einzelne Bildschirmseite einer Android-App), die miteinander kommunizieren und in ihrer 'Lebenszeit' ein vorgegebenes Zustandsmodell \ref{figure:androidZustandsmodell} durchlaufen.

\begin{figure}[htp]
	\centering
  	\includegraphics[width=0.3\textwidth]{img/modelle/AndroidZustandsmodell.png}
	\caption[Android Zustandsmodell]{Android Zustandsmodell\protect\footnotemark}
	\label{figure:androidZustandsmodell}
\end{figure}
\footnotetext{http://www.javatpoint.com/images/androidimages/Android-Activity-Lifecycle.png}

Dieses Zustandsmodell ist auch anfangs einer der Nachteile von Android, da es nicht so leicht zu verstehen ist und uns einige Probleme bereitet hat. Nachdem wir uns aber im Laufe der App-Entwicklung immer mehr mit Android vertraut gemacht haben, war auch das Modell kein Problem mehr, sondern im Gegenteil sehr angenehm, da es sehr logisch und gut durchdacht ist. Eine weitere Schwierigkeit, die während der Entwicklung auftrat, ist die Vielzahl an verschiedenen Android-Versionen und Geräten. Da wir unsere App für so viele Versionen wie möglichen entwickeln wollten, kamen auch einige Probleme auf. Beispielsweise sind manche Libraries oder Frameworks erst ab einer bestimmten Android-Version verfügbar, und die vielen verschiedenen Geräte haben meist unterschiedliche Displaygrößen und Seitenverhältnisse.\\
Die Vorteile von Android überwiegen aber unserer Meinung nach, vor allem, nachdem man sich damit tiefer beschäftigt und sich eingearbeitet hat. Einer der größten Vorteile ist die sehr gute Dokumentation von Android, wodurch das Einlesen in die Möglichkeiten und Funktionen recht leicht ist. Auch die weltweite Verbreitung und Beliebtheit von Android ist hier ein Vorteil, da es sehr viele Entwickler gibt und so jedes Problem schon einmal aufgetreten ist und daher auch meist schon Lösungen oder Hilfestellungen verfügbar sind.

Außerdem wird Android stetig Weiterentwickelt, weshalb immer mehr möglich ist und der Umgang mit bestimmten Funktionen, wie etwa der Zugriff auf Gerätefunktionen wie Kamera oder Galerie, immer leichter wird. Weitere Vorteile für uns sind die Vertrautheit mit Java und die Einfachheit der eigens von Google bereitgestellten Entwicklungsumgebung, ``Android Studio''.

% Abschnitt: Frameworks
\section{Frameworks}
\label{sec:grundlagen:frameworks}
Frameworks sind Ansammlungen von Funktionen, die wiederverwendet werden können um einen gezielten Bereich der Implementierung zu erleichtern und die Anzahl an neu zu implementierenden Funktionen zu verringern.\\
Wir haben in unserer App drei Frameworks als Hilfen genutzt:
\begin{itemize}
\item Picasso\footnote{http://square.github.io/picasso/}: Erlaubt einen wesentlich einfacheren Umgang hinsichtlich Anzeige, (asynchronem) Laden, und Transformation von Bildern.
\item MPAndroidCharts\footnote{https://github.com/PhilJay/MPAndroidChart}: Ermöglicht die Erstellung von Diagrammen - in unserem Fall Kuchendiagramme zur Visualisierung der Statistiken
\item Floating Action Button / Floating Action Menu\footnote{https://github.com/Clans/FloatingActionButton}: Eine Erweiterung des standardmäßigen Floating Action Buttons in Form eines Menüs, das ein- und ausgeklappt werden kann.
\end{itemize}




















\newcommand{\reqtable}[3]{
    \begin{center}
    \begin{tabular}{ | l | p{13cm} |}
    \hline
    \textbf{ID:} & \textbf{#1} \\ \hline
    TITEL: & #2 \\ \hline
    BES: & #3 \\
    \hline
    \end{tabular}
    \end{center}
}

\chapter{Anforderungsanalyse}	
\label{cha:anforderungsanalyse}
In diesem Kapitel werden die ursprünglich definierten funktionalen und nichtfunktionalen Anforderungen tabellarisch dargestellt. Jede der Anforderungen hat einen eindeutigen Identifikator (ID), einen Titel (TITEL), und eine Beschreibung (BES).

% Abschnitt: Funktionale Anforderungen
\section{Funktionale Anforderungen}
\label{sec:anforderungsanalyse:funktional}

\reqtable{FA1}{Startmenü}{Nach dem Start der Anwendung sieht der Benutzer ein Startmenü mit den Einträgen „Spiel starten“, „Einstellungen“, „Spielregeln“, „Deckübersicht“ („Rangliste“), („Infos“)}
\reqtable{FA2}{Einstellungen}{Es gibt eine Möglichkeit, die Spieleinstellungen zu ändern. Diese umfassen Schwierigkeitsgrad, Soundeffekte (an/aus) und Spielmodus (rundenbasiert/zeitbasiert/Kartenbasiert)}
\reqtable{FA3}{Spielregeln}{Es gibt eine Möglichkeit, die Spielregeln anzuzeigen.}
\reqtable{FA4}{Infoseite}{Es gibt eine Möglichkeit, eine Infoseite (mit Copyright- und weiteren Informationen) anzuzeigen}
\reqtable{FA5}{Deckübersicht}{Der Benutzer hat die Möglichkeit eine Liste mit allen verfügbaren Decks anzuzeigen. Beim Auswählen eines Decks kann der Benutzer durch die Karten des Decks scrollen. Dabei sieht er bei jeder Karte das Bild und die Attribute. Außerdem kann der Benutzer auf der Deckübersichts-Seite neue Decks hinzufügen. Diese Decks dienen als „Erweiterung“ und der Benutzer kann sie sich herunterladen.}
\reqtable{FA6}{Spiel starten}{Der Benutzer hat vom Startmenü aus die Möglichkeit, das Spiel zu starten. Dafür öffnet sich ein Dialog, auf dem der Benutzer das Deck auswählt und festlegt, ob er gegen einen anderen Spieler oder gegen einen Computer spielen will. Danach werden die Karten auf die beiden Spieler verteilt und es wird zufällig bestimmt, welcher Spieler beginnt.}
\reqtable{FA7}{Spielablauf}{Während des Spiels sieht der Spieler pro Runde nur seine „oberste Karte im Stapel“. In einer Runde werden die Werte des erstgewählten Attributs verglichen. Nach Auswahl eines Attributs vom ersten Spieler werden die zu vergleichenden Werte beider Spieler angezeigt und das gewinnende Attribut markiert. Bei einem Gleichstand werden die Karten beider Spieler unter den jeweils eigenen Stapel gelegt.}
\reqtable{FA8}{Spielstand anzeigen}{Während des Spiels wird dauerhaft der Spielstand (Anzahl Karten Spieler 1 : Anzahl Karten Spieler 2) angezeigt.}
\reqtable{FA9}{Spielzeit anzeigen}{Beim zeitbasierten Modus wird neben dem Spielstand auch die verbleibende Rundenzeit und die verbleibende Gesamt-Spielzeit angezeigt.}
\reqtable{FA10}{Spielende}{Das Spiel ist vorbei, wenn
\begin{itemize}
\item {[alle Modi]} ein Spieler alle Karten hat (Dieser Spieler hat gewonnen)
\item {[Zeitbasiert]} die Zeit abgelaufen ist (Der Spieler mit den meisten Karten hat gewonnen)
\item {[Rundenbasiert]} die vorher ausgewählte Anzahl an Runden gespielt worden sind (Spieler mit den meisten Karten hat gewonnen)
\end{itemize}
}
\reqtable{FA11}{Speicherung der Daten}{Die Daten des Spiels (Decks, Karten und ihre Attribute) werden in einer Datenbank gespeichert. Dabei gilt für jedes Deck:
\begin{itemize}
\item Die Anzahl an Karten soll eine gerade Zahl zwischen 16 und 64 sein.
\item Die Anzahl an Attributen soll eine Zahl zwischen 5 und 10 sein.
\item Für jedes Attribut wird spezifiziert, ob ein höherer Wert oder ein niedriger Wert gewinnt.
\end{itemize}
}

% Abschnitt: Nichtfunktionale Anforderungen
\section{Nichtfunktionale Anforderungen}
\label{sec:anforderungsanalyse:nichtfunktional}

\reqtable{NFA1}{Entwicklungssprache und -Umgebung}{Die Anwendung wird in Java mit der Entwicklungsumgebung „Android Studio“ entwickelt und soll auf Geräten mit Betriebssystemen ab Android [4.1] funktionieren.}
\reqtable{NFA2}{Reaktionszeit}{Die Reaktionszeit der Anwendung soll zu jeder Zeit maximal 1,5 Sekunden betragen}
\reqtable{NFA3}{Persistenter Spielzustand}{Während eines Spiels soll die App minimiert werden können (z.B. durch den Home Button), ohne dass der Spielzustand verloren geht. D.h., ein Spiel soll bei erneutem Öffnen fortgesetzt werden können.}
\reqtable{NFA4}{Benutzbarkeit}{Die Anwendung soll intuitiv bedienbar sein. D.h., die Buttons und andere Auswahlmöglichkeiten sollen eine ausreichende Größe haben und wie erwartet reagieren.}
\reqtable{NFA5}{Offline-Modus}{Die Anwendung soll stets auch das Spielen ohne Internetanbindung ermöglichen.}
%\chapter{Konzept und Entwurf}

Wie üblich bei einem Projekt startet man mit Mockups der Anwendung, damit der Kunde eine genauere Vorstellung seiner Ideen bekommt. 

Wir haben uns ziemlich stark an unsere Mockups gehalten und im Folgenden sieht man jedes Mockup gepaart mit dem dazugehörigen Screenshot aus der fertigen App.

Wenn man die App öffnet befindet man sich als erstes auf der Startseite.\\

\begin{figure}[h]
    \centering
    \begin{minipage}{0.45\textwidth}
        \centering
        \includegraphics[width=0.75\textwidth]{img/mockups/main_screen.png}
        \caption{first figure}
    \end{minipage}
    \begin{minipage}{0.45\textwidth}
        \centering
        \includegraphics[width=0.75\textwidth]{img/screenshots/device_main_screen.png}
        \caption{second figure}
    \end{minipage}
\end{figure}

Als nächstes manövriert man zu einem neuen Spiel. Hier haben wir, anders als in dem Mockup, die Einstellungen nur angezeigt, aber man kann sie natürlch noch ändern bevor man das Spiel startet. Ein Deck muss aber jedes mal gewählt werden. \\

\begin{figure}[h]
    \centering
    \begin{minipage}{0.45\textwidth}
        \centering
        \includegraphics[width=0.4\textwidth]{img/mockups/neues_spiel.png}
        \caption{first figure}
    \end{minipage}\hfill
    \begin{minipage}{0.45\textwidth}
        \centering
        \includegraphics[width=0.4\textwidth]{img/screenshots/device_new_game.png}
        \caption{second figure}
    \end{minipage}
\end{figure}

Will man jetzt doch noch etwas an den Einstellungen ändern, tippt man auf Einstellungen anpassen und kann nun beliebig Anpassungen vornehmen.

\begin{figure}[h]
    \centering
    \begin{minipage}{0.45\textwidth}
        \centering
        \includegraphics[width=0.4\textwidth]{img/mockups/einstellungen.png}
        \caption{first figure}
    \end{minipage}\hfill
    \begin{minipage}{0.45\textwidth}
        \centering
        \includegraphics[width=0.4\textwidth]{img/screenshots/device_settings.png}
        \caption{second figure}
    \end{minipage}
\end{figure}

\chapter{Konzept, Entwurf und Implementierung}
\label{cha:implementierung}

Wir erklären in diesem Kapitel die einzelnen Funktionen unserer App genauer, unterstützen sie an Hand von Screenshots und zeigen Vergleichend auch die Mockups die zu Beginn des Projekts erstellt wurden, um dem Kunden seine Vorstellungen visualisiert präsentieren zu können.

% Abschnitt: Implementierungsdetails
\section{Implementierungsdetails}
\label{sec:implementierung:implementierungsdetails}

Nach dem öffnen der App befindet man sich zu Beginn im Hauptmenü, welches in Abbildung \ref{figure:implementierungmenue} zu sehen ist. Für das Hauptmenü wir, wie auch für die gesamte App, haben wir ein ansprechendes Farbklima gewählt.\\

\begin{figure}[h]
    \centering
    \begin{minipage}{0.49\textwidth}
        \centering
        \includegraphics[width=0.4\textwidth]{img/screenshots/device_main_screen.png}
        \caption{Hauptmenü der App}
		\label{figure:implementierungmenue}
    \end{minipage}
    \begin{minipage}{0.49\textwidth}
        \centering
        \includegraphics[width=0.4\textwidth]{img/mockups/main_screen.png}
        \caption{Mockup Hauptmenü}
    \end{minipage}
\end{figure}


Durch den Menüpunkt 'SPIELEN' landet der Spieler auf einem Bildschirm, in welchem er ein Deck wählen kann und bei Bedarf die Einstellungen anpassen kann, wie in Abbildung \ref{figure:implementierungeinstellungen} zu sehen ist. Diese sind frei wählbar und alle Kombinationen sind möglich. Die Einstellungen werden lokal auf dem Smartphone gespeichert und können beim nächstmaligen Starten direkt übernommen werden, ohne sie jedes mal neu zu ändern. Das ermöglicht einen schnellen Start ins Spiel.\\

\begin{figure}[h]
    \centering
    \begin{minipage}{0.49\textwidth}
        \centering
        \includegraphics[width=0.4\textwidth]{img/screenshots/device_settings.png}
		\caption{Einstellungsmenü der App}
		\label{figure:implementierungeinstellungen}
    \end{minipage}
    \begin{minipage}{0.49\textwidth}
        \centering
        \includegraphics[width=0.4\textwidth]{img/mockups/einstellungen.png}
        \caption{Mockup Einstellungen}
    \end{minipage}
\end{figure}

Nun ist das eigentliche Spiel gestartet. Der Spieler sieht jeweils nur seine erste oberste Karte, wie in Abbildung \ref{figure:implementierungspiel1} abgebildet ist. Für jedes Attribut wird neben dem Wert (beziehungsweise einem Fragezeichen im ``Expertenmodus'') auch die jeweilige Einheit und die Siegesvariante (höher oder niedriger gewinnt) angegeben. Zwischen den Bildern einer Karte kann durch Swipe-Gesten gewechselt werden und zu jedem Bild kann, falls vorhanden, die passende Information durch drücken auf den Info-Buttonn angezeigt werden. Wenn der Spieler am Zug ist, sind seine Attribute aktiv und er kann ein Attribut für den Vergleich auswählen. Für seinen Zug hat der Spieler unbegrenzt Zeit, außer im Zeitmodus, in welchem die Zeit für einen Zug auf 30 Sekunden beschränkt ist, damit der Spieler sich nicht auf seinen bisherigen Erfolgen ausruhen kann. Die Zeit für seinen Zug wird ihm angezeigt und zudem werden die letzten 10 Sekunden durch ein akustisches Signal verdeutlicht, sofern in den Einstellungen die Sounds aktiviert sind. In allen Spielvarianten wird dem Spieler während des Spiels der aktuelle Zwischenstand (durch Punkte oder Karten) und die verbliebende Spielzeit(verbleibende Karten/Punkte/Zeit) angezeigt.\\

\begin{figure}[h]
    \centering
    \begin{minipage}{0.49\textwidth}
        \centering
        \includegraphics[width=0.4\textwidth]{img/screenshots/device_select_attr.png}
		\caption{Kartenansicht im Spiel}
		\label{figure:implementierungspiel1}
    \end{minipage}
    \begin{minipage}{0.49\textwidth}
        \centering
        \includegraphics[width=0.4\textwidth]{img/mockups/spiel_attributauswahl.png}
        \caption{Mockup Karte im Spiel}
    \end{minipage}
\end{figure}


Ist der Gegner am Zug, sieht die Kartenansicht des Spielers gleich aus, aber seine Attribute sind inaktiv und nicht auswählbar. Nun muss der Spieler eine gewisse Zeit warten, bis der Computer sein Attribut für den Vergleich ausgewählt hat. Diese Zeit kann er nutzen, um seine Karte zu betrachten. Im Schwierigkeitsgrad ``leicht'' wählt der Computer ein zufälliges Attribut für den Vergleich aus. Im Schwierigkeidsgrad ``mittel'' wählt er unter einer zufälligen Menge an Attributen, welche ungefähr die Mächtigkeit der Hälfte aller Attribute hat, den besten Wert aus. Im Schwierigkeitsgrad ``schwer'' wählt der Computer unter allen Attributen zu einer gewissen Wahrscheinlichkeit den besten Wert aus. Diese Schwierigkeitsgrade sorgen dafür, dass der Spieler zwar einen Anstieg der Schwierigkeit des Computers spürt, jedoch immer eine Chance zu gewinnen hat. Zudem bliebt durch die Zufallswerte der Zug des Computergegners unvorhersehbar.

Nach jedem gespielten Zug sieht der Spieler den Vergleichsbildschirm mit seiner Karte und der Karte seines Gegners, wie in Abbildung \ref{figure:implementierungspiel2}  dargestellt (Wir sind hier insofern vom Mockup abgewichen, dass das Bild einer Karte mit dem gewählten Attribut angezeigt wird, statt nur alle Attibute anzuzeigen, da es so attraktiver aussieht). Dort kann wieder bei beiden Karten zwischen den Bildern durch Swipe-Bewegungen gewechselt werden und die jeweiligen Informationen durch Drücken des Info-Buttons angesehen werden. Der gewählte Wert beider Spieler wird angezeigt und der Gewinner hervorgehoben. Der Gewinner bekommt beide Karten, bei einem Unentschieden behält jeder Spieler seine Karte. Zudem wird der Punktestand aktualisiert. Im Kartenmodus und im Zeitmodus ist jede Karte genau ein Punkt wert. Im Punktemodus berechnen sich die Punkte für den Gewinner durch den prozentualen Unterschied beider Werte. So kann der Spieler mit ein und dem selben Attribut einer Karte unterschiedlich viele Punkte machen, je nachdem welche Karte der Gegner gerade besitzt. Außerdem ist es dadurch möglich, dss ein Spieler mit weniger Karten als der Gegner gewinnt, weil er durch geschickte Spielzüge mehr Punkte gesammelt hat.\\

\begin{figure}[h]
    \centering
    \begin{minipage}{0.49\textwidth}
        \centering
        \includegraphics[width=0.4\textwidth]{img/screenshots/device_comparison.png}
		\caption{Kartenvergleich}
		\label{figure:implementierungspiel2}    
	\end{minipage}
    \begin{minipage}{0.49\textwidth}
        \centering
        \includegraphics[width=0.4\textwidth]{img/mockups/spiel_vergleich.png}
        \caption{Mockup Kartenvergleich}
    \end{minipage}
\end{figure}


Der Spieler hat jederzeit die Möglichkeit, das laufende Spiel zu unterbrechen. Dazu wird ein Spiel automatisch zwischengespeichert, wenn der Spieler das laufende Spiel oder die App verlässt. Dieses kann der Spieler zu einem späteren Zeitpunkt fortsetzen. Wenn er ein neues Spiel beginnen möchte wird er gefragt, ob er lieber das alte Spiel fortsetzen will oder ein neues beginnen und das alte überschreiben möchte.

Steht ein Gewinner fest, wird der Spielend-Bildschirm angezeigt. Auf diesem steht der Gewinner, der Endstand und die gesammelten Punkte des Gewinners, wie in Abbildung \ref{figure:implementierungspielende} zu sehen.\\

\begin{figure}[h]
    \centering
    \begin{minipage}{0.49\textwidth}
        \centering
        \includegraphics[width=0.4\textwidth]{img/screenshots/device_game_end.png}
		\caption{Endscreen der App}
		\label{figure:implementierungspielende}   
	\end{minipage}
    \begin{minipage}{0.49\textwidth}
        \centering
        \includegraphics[width=0.4\textwidth]{img/mockups/spiel_ende.png}
        \caption{Mockup Endscreen}
    \end{minipage}
\end{figure}

Ist der Spieler der Gewinner, wird anhand dieser Punkte entschieden, ob der Spieler es in die Rangliste geschafft hat, und falls ja, auf welche Position. Die Punkte berechnen sich durch eine Kombination aus den gesammelten Punkten im Spiel, dem Schwierigkeitsgrad des Spiels, dem Spielmodus (normal, Insanemodus oder Expertenmodus) und der Anzahl der eingestellten Runden/Zeit/Punkte. Dadurch bleibt die Rangliste immer fair und kann nicht durch vereinfachte Einstellungen oder eine erhöhte Anzahl an Spielrunden beeinflusst werden. Der Spieler kann sich hier direkt mit seinem Namen in die Rangliste eintragen, wobei der jeweils letzte Name gespeichert wird. Außerdem kann er von diesem Bildschirm direkt die Ranglisten (Abbildung \ref{figure:implementierungrangliste}) einsehen oder eine Revanche starten, welche ein Spiel mit dem gleichen Deck und den gleichen Einstellungen startet.\\

\begin{figure}[h]
    \centering
    \begin{minipage}{0.49\textwidth}
        \centering
        \includegraphics[width=0.4\textwidth]{img/screenshots/device_high_scores.png}
		\caption{Rangliste der App}
		\label{figure:implementierungrangliste}   
	\end{minipage}
    \begin{minipage}{0.49\textwidth}
        \centering
        \includegraphics[width=0.4\textwidth]{img/mockups/rangliste.png}
        \caption{Mockup Rangliste}
    \end{minipage}
\end{figure}

Neben dem Spiel ansich befinden sich die meisten Implementierungsdetails unserer Anwendung im Menüpunkt 'GALERIE'. Dort hat der Spieler eine Übersicht aller Decks, die auf dem Smartphone vorhanden sind. Dieses Menü wurde durch ein Grid-Layout-Adapter erstellt und findet sich an vielen Stellen in unserer App wieder. Es erlaubt neben einer einfachen und immer gleich großen und gleich formartierten Darstellung der Bilder auch ein einfaches Scrollen und Anzeigen von Deckinformationen durch drücken auf den Info-Button. Ein Bild davon gibt es in Abbildung \ref{figure:implementierunggalerie} Im Menü kann der Spieler die Karten eines Decks angucken, was ungefähr ähnlich ausieht wie die Kartenansicht während eines Spiels in Abbildung \ref{figure:implementierungspiel1}. Zudem kann er hier die Namen der Decks umbenennen, die einzelnen Werte und Bilder mit dazugehörigen Beschreibungen einzelner Karten der Decks ändern sowie neue Karten zu Decks hinzufügen oder Karten löschen. Außerdem kann ein Deck auch gelöscht werden oder über den Deckcreator ein völlig neues Deck mit neuen Attributen erstellt werden. Neue Decks können dann auf den Deckstore hochgeladen werden, sodass es für andere Nutzer zur Verfügung steht. Über diesen können wiederum auch neue Decks von anderen Nutzern auf das Smartphone herunter geladen werden. Auf diese zuletzt genannten Funktionen werden wir in den Besonderheiten genauer eingehen.\\

\begin{figure}[h]
    \centering
    \begin{minipage}{0.49\textwidth}
        \centering
        \includegraphics[width=0.4\textwidth]{img/screenshots/device_gallery.png}
		\caption{Die Galerie der App}
		\label{figure:implementierunggalerie} 
	\end{minipage}
    \begin{minipage}{0.49\textwidth}
        \centering
        \includegraphics[width=0.4\textwidth]{img/mockups/galerie_uebersicht.png}
        \caption{Mockup Galerie}
    \end{minipage}
\end{figure}

Neben der Rangliste gibt es auch noch einige Statistiken in unserer App, wie zum Beispiel die Anzahl der gespielten Spiele und die dabei gewonnen und verlorenen Spiele für die verschiedenen Spielmodi. Diese werden mit Hilfe der Bibliothek MPAndroidCharts als Piechart dargestellt und werden nach jedem bis zum Ende durchgespielten Spiel aktualisiert. Ein Bild davon gibt es in Abbildung \ref{figure:implementierungstatistiken}.\\

\begin{figure}[h]
    \centering
    \begin{minipage}{0.49\textwidth}
        \centering
        \includegraphics[width=0.4\textwidth]{img/screenshots/device_statistics.png}
		\caption{Die Statistiken der App}
		\label{figure:implementierungstatistiken}
	\end{minipage}
    \begin{minipage}{0.49\textwidth}
        \centering
        \includegraphics[width=0.4\textwidth]{img/mockups/statistiken.png}
        \caption{Mockup Statistiken}
    \end{minipage}
\end{figure}



% Abschnitt: Architektur
\section{Architektur}
\label{sec:implementierung:architektur}

TODO

- ein paar Einleitungssätze
- Datenmodell aus Präsentation kopieren und erlüutern, was besonders ist (auf Speicherung mit JSON eingehen)
- Klassen-/Activity-Modell aus Präsentation kopieren
- vielleicht ein paar Sätze zu allgemeinem Vorgehen und Aufteilung??

% Abschnitt: Besonderheiten 
\section{Besonderheiten}
\label{sec:implementierung:besonderheiten }

Im Vergleich zu anderen Quartett Apps gibt es bei unserer Quartett42 App, neben den verschiedenen Spielmodi und Decks, besonders zwei Funktionen, die so bisher noch nicht verfügbar waren. Diese sind zum einen unser Deckcreator und -editor und zum anderen der Deck Down- und Upload.	

\subsection{Deckcreator und -editor}
\label{sec:implementierung:besonderheiten:deckcreator }

\begin{figure}[htp]
	\centering
  	\includegraphics[width=0.3\textwidth]{img/screenshots/device_new_deck.png}
	\caption{Erster Schritt im Deckcreator}
	\label{figure:implementierungdeckcreator}
\end{figure}

Der Deckcreator erlaubt das Erstellen neuer Decks oder das Bearbeiten vorhandener Decks. Dadurch kann jeder Spieler ein komplett neues eigenes Deck nach seinen Wünschen erstellen. Über die Galerie gelangt der Spieler in den Deckcreator. Dort muss er erst einmal allgemeine Angaben zu seinem gewünschten Deck angeben, wie in Abbildung \ref{figure:implementierungdeckcreator} dargestellt. Dort mus ein Namen für das Deck eingegeben werden, wobei der Creator ein Deck nur erstellt, wenn dieser Name nicht bereits vergeben ist. Beschreibung und Bild sind optional, wobei bei keinem angegebenem Bild ein Standardbild verwendet wird. Das Bild kann entweder aus der Fotogalerie des Smartphones gewählt werden oder direkt aufgenommen werden, wobei es in beiden Fällen direkt auf eine akzeptable Größe verkleinert wird. Zudem müssen eine beliebige Anzahl an Attributen festgelegt werden, und für jede Attribut, wann es gewinnt, und welche Einheit es hat. Hier müssen verschiedene Überprüfungen statt finden. So darf kein Attribut zwei mal verwendet werden und es dürfen nirgends unerlaubte Zeichen oder leere Eingaben vorkommen. Sind alle Eingaben korrekt, wird das Deck zwischen gespeichert und der Nutzer zum zweiten Schritt weiter geleitet.

\begin{figure}[htp]
	\centering
  	\includegraphics[width=0.3\textwidth]{img/screenshots/device_new_card.png}
	\caption{Zweiter Schritt im Deckcreator}
	\label{figure:implementierungdeckeditor}
\end{figure}

Im zweiten Schritt geht es um die Erstellung einzelner Karten und deren Werte. Wenn der Benutzer ein bereits vorhandenes Deck bearbeiten will, gelangt er direkt in diesen Schritt. Ein Bild davon kann man in Abbildung \ref{figure:implementierungdeckeditor} sehen. Für jede Karte muss ein Name eingegeben werden, welcher pro Deck wieder nur einmal vorkommen darf. Zudem kann eine beliebige Anzahl an Bildern wie beim Deckbild und dazu passende Beschreibungen angegeben werden. Für jedes Attribut der Karte muss ein Wert angegeben werden. Auch hierbei muss wieder alles auf gültige Eingaben überprüft werden und es dürfen keine leeren Eingaben vorkommen. Der Spieler kann jederzeit eine neue Karte hinzufügen und eine alte Karte löschen. Zudem kann er zwischen allen Karten hin und her wechseln. Dabei wird das Deck bei jedem Wechsel zwischengespeichert. Damit ein Deck spielbar ist, muss es mindestens zwei Karten besitzen. Ist der Nutzer fertig mit dem Erstellen oder Bearbeiten eines Decks, kann er es speichern und ansehen oder spielen. Zudem besteht die Möglichkeit, den Erstellvorgang zu unterbrechen und zu einer anderen Zeit fortzusetzen.

\subsection{Down- und Upload von Decks}
\label{sec:implementierung:besonderheiten:deckdownload }

Über den Deckstore können neue Decks runter und hoch geladen werden. Das ermöglicht das Teilen von selbst erstellten Decks mit anderen und führt dazu, dass immer wieder neue Decks herunter geladen werden können. Die Speicherung der Decks findet auf einem Server des Institutes für Datenbank und Informationssysteme der Universität Ulm statt. Die Daten Decks werden dabei als JSON-Strings gespeichert. 

Über die Galerie gelangt man in den Deckstore, in welchem dem Spieler alle Decks aus dem Deckstore angezeigt werden, welche er noch nicht selbst auf sein Smartphone herunter geladen hat. Diese erfragt die Anwendung durch einen Http-Request an den Server, welcher eine Liste alle auf dem Server vorhandenen Decks zurück liefert. Zudem wird für jedes Deck noch das Deckbild und die Deckbeschreibung herunter geladen und gecached. Außerdem erhält der Nutzer beim Starten der App eine Notifikation in der Statusleiste, falls neue Decks im Decksotre verfügbar sind. Die Darstellung im Deckstore erfolgt wieder über den Grid-View-Adapter, ähnlich zu Abbildung \ref{figure:implementierunggalerie}. Der Benutzer kann nun ein Deck auswählen, welches er herunterladen möchte. Der Nutzer sieht während dem gesamten Ladevorgang einen Fortschrittsbalken, welcher den Fortschritt in Prozent berechnet und darstellt. Für das Deck wird dann zuerst die allgemeine Deckinformation heruntergeladen sowie eine Liste aller Karten. Dann wird jede einzelne Karte heruntergeladen und für jede Karte alle dazugehörigen Bilder und Werte. Für jede dieser Daten wird ein Http-Request an den Server gesendet, welcher die Daten in Form eines JSON-Strings zurück gibt. Da nicht nur unsere Anwendung sondern mehrere Anwendungen Decks auf den Server hochladen können, welche nicht unbedingt formal korrekt sein können, wird in jedem Schritt überprüft, ob das Deck alle Anforderungen erfüllt. Dazu gehören zum Beispiel eine Mindestanzahl an Karten und Attributen, keine unerlaubten Zeichen und leere Angaben sowie keine falschen Bilddateien. Letzteres wird durch ein Standardbild ersetzt, bei allen anderen Fehlern wird der Download abgebrochen, um unnötige Wartezeiten zu vermeiden. Bilder, die zu groß sind, werden automatisch verkleinert. Nach dem vollständigen Download aller Daten und der Kontrolle wird das Deck zusammen gebaut und in den lokalen Speicher gespeichert. Hierbei wird bevozugt versucht, die Daten auf einem externen Speicher zu speichern, um nicht den internen speicher zu belasten. Eine Grafik, wie der Download funktioniert, sieht man in Abbildung \ref{figure:implementierungdownundupload}.

In der Galerie hat der Benutzer die Möglichkeit, ein Deck direkt auf den Server hochzuladen. Zuerst wird durch eine Anfrage an den Server überprüft, ob das Deck schon im Deckstore vorhanden ist. Ist dies der Fall, kann das selbe Deck nicht erneut hochgeladen werden. Fall nicht, bekommt der Benutzer wieder eine Ladeanzeige mit Fortschrittsbalken zu sehen. Bevor das Deck hochgeladen wird, wird auch dieses zur Sicherheit auf Fehler überprüft, welche aber im Normalfall dank der Fehlerüberprüfung bei der Deckerstellung nicht vorkommen dürfen. Dann wird das Deck in viele einzelne JSON-Strings zerlegt und dem Server über einen Http-Request das Deck mit seinen Informationen bekannt gegeben. Danach wird wie beim Download, jede Karte und mit ihm seine Werte Werte und Bilder als JSON-String über einen extra Http-Request gesendet, wobei die Bilder in einen Base64-String konvertiert und versendet werden. Nach dem vollständigen Upload ist das Deck für alle Benutzer über den Deckstore verfügbar. Eine Grafik, wie der Upload funktioniert, sieht man in Abbildung \ref{figure:implementierungdownundupload}. Sollte der Benutzer keine Internetverbindung, wird sowohl der Download als auch der Upload abgebrochen.
\newpage

%Download- und Upload-Verlauf-Modell
\begin{figure}[htp]
\centering
\begin{minipage}{.45\textwidth}
  \centering
  \includegraphics[width=.95\linewidth]{img/modelle/downloadmodell.png}
\end{minipage}%
\begin{minipage}{.45\textwidth}
  \centering
  \includegraphics[width=.95\linewidth]{img/modelle/uploadmodell.png}
\end{minipage}%
\caption{Links: Downloadmodell, rechts: Uploadmodell }
\label{figure:implementierungdownundupload}
\end{figure}
\newpage

% Abschnitt: Schwierigkeiten während der Implementierung 
\section{Schwierigkeiten während der Implementierung}
\label{sec:implementierung:schwierigkeiten }	

TODO

Wie in Präsentation
\chapter{Anforderungsabgleich}
\label{cha:anforderungsabgleich}

% Abschnitt: Funktionale Anforderungen
\section{Funktionale Anforderungen}
\label{sec:anforderungsabgleich:funktional}


% Abschnitt: Nicht Funktionale Anforderungen
\section{Nicht Funktionale Anforderungen}
\label{sec:anforderungsabgleich:nichtfunktional}
\chapter{Zusammenfassung und Ausblick}
\label{cha:zusammenfassungUndAusblick}

In diesem Kapitel soll ein kurzes Fazit zur App-Entwicklung und zum Anwendungsfach allgemein festgehalten werden. 

Auch wenn der Einstieg in die App-Entwicklung mit dem ``EiTimer'' nicht ganz leicht war, haben wir dort schon gemerkt, dass uns die Entwicklung liegt und Spaß bereitet. Wir haben gerade zu Beginn bei der Implementierung der ``EiTimer''-Anwendung und später unserer Quartett42-App vieles gelernt, was wir im zweiten Teil des Anwendungsfachs natürlich anwenden und weiter vertiefen können.

In unseren Augen ist uns die Anwendung größtenteils gut gelungen. Die Qualität der Anwendung wäre theoretisch ausreichend für eine Veröffentlichung im App Store, da es gerade bei Quartett-Spielen eine Marktlücke gibt.\\ Natürlich gäbe es noch einige Stellen, an der sie weiter verbessert werden könnte, wie beispielsweise das Design oder weitere naheliegende Funktionen (Spieler vs. Spieler, Online/Bluetooth-Spiel, ...).\\ Ein nicht zu unterschätzender Faktor war hierbei die Zeit, die neben den anderen Vorlesungen, Prüfungen, etc. limitiert war. Uns ist jedoch im gegebenen Zeitrahmen ein abgerundetes Produkt ohne ``unfertige'' Funktionen gelungen.

Da wir beim ersten Teil des Projekts gut als Team zusammenarbeiten konnten, haben wir beschlossen auch im zweiten Teil des Anwendungsfachs wieder gemeinsam eine Anwendung zu entwickeln, für die wir auch schon einige Ideen gesammelt haben. Im zweiten Teil des Projekts werden wir versuchen, weitere Wissenslücken in der Android-Entwicklung zu schließen.

% Bibliograhpy
\bibliographystyle{splncs}
\begingroup
\interlinepenalty 10000
\sloppy
\bibliography{literature}
\endgroup

% anhänge
%\appendix
% hier kommen die anhänge
%\chapter{Quelltexte}

In diesem Anhang sind einige wichtige Quelltexte aufgeführt.

\begin{lstlisting}[caption={Zeilencode}]
public class Hello {
    public static void main(String[] args) {
        System.out.println("Hello World");
    }
}
\end{lstlisting}


\backmatter			% abtrennung für verzeichnisse

% hier die verzeichnisse
\listoffigures
%\listoftables


\clearpage
%\erklaerung

\end{document}
